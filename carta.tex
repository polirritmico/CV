%!TEX TS-program = xelatex
%!TEX encoding = UTF-8 Unicode

%-------------------------------------------------------------------------------
% CONFIGURATIONS
%-------------------------------------------------------------------------------
% 'a4paper' size by default, use 'letterpaper' for US letter
\documentclass[11pt, a4paper]{awesome-cv}

% Español
\usepackage[spanish]{babel}

% Configure page margins with geometry
\geometry{left=1.4cm, top=.8cm, right=1.4cm, bottom=1.8cm, footskip=.5cm}

% Specify the location of the included fonts
\fontdir[fonts/]

% If you would like to change the social information separator from a
% pipe (|) to something else replace \textbar
\renewcommand{\acvHeaderSocialSep}{\quad\textbar\quad}

%---------------------------------------

% Color for highlights
% Awesome Colors: awesome-emerald, awesome-skyblue, awesome-red,
%                 awesome-pink, awesome-orange, awesome-nephritis,
%                 awesome-concrete, awesome-darknight
\colorlet{awesome}{awesome-red}
% Uncomment if you would like to specify your own color
%\definecolor{awesome}{HTML}{CA63A8}

% Set false if you don't want to highlight section with awesome color
\setbool{acvSectionColorHighlight}{true}

% Colors for text
% Uncomment if you would like to specify your own color
%\definecolor{darktext}{HTML}{414141}
%\definecolor{text}{HTML}{333333}
%\definecolor{graytext}{HTML}{5D5D5D}
%\definecolor{lighttext}{HTML}{999999}


%-------------------------------------------------------------------------------
%	                                   HEADER
%-------------------------------------------------------------------------------
\input{sections/header-carta}


%-------------------------------------------------------------------------------
%                              LETTER INFORMATION
%-------------------------------------------------------------------------------
%	All of the below lines must be filled out
%-------------------------------------------------------------------------------

% The company being applied to
\recipient
  {Equipo de recursos humanos}
  {Nombre compañía.\\Dirección\\Localidad, CP}
% The date on the letter, default is the date of compilation
\letterdate{\today}
% The title of the letter
\lettertitle{Solicitud de empleo para CARGO}
% How the letter is opened
\letteropening{Estimado Sr. Apellido,}
% How the letter is closed
\letterclosing{Atentamente,}
% Any enclosures with the letter
\letterenclosure[Adjunto]{Curriculum vitae}


%-------------------------------------------------------------------------------
%	                                  DOCUMENT
%-------------------------------------------------------------------------------
\begin{document}

% Add the HEADER. optional argument to change alignment [C|L|R]
\makecvheader[R]

% Footer with 3 arguments ({L} {C} {R}). Leave blank if not needed
\makecvfooter
  {}
  {Eduardo Bray G.~~~·~~~Carta de presentación}
  {}

% Print the title with above letter informations
\makelettertitle

%-------------------------------------------------------------------------------
%                               CONTENIDO DE LA CARTA
%-------------------------------------------------------------------------------
\begin{cvletter}

% 1. [ ] Razón por la que escribo.
% 2. [ ] Cómo me enteré del cargo o la organización.
% 3. [ ] Añadir información básica de mí.

Junto con saludar escribo esta carta en respuesta al anuncio publicado en
\textbf{Linkedin} para postular al cargo de \textbf{Programador}. Pese a mi
background en el área artística, siempre he dado soporte y desarrollado
herramientas tecnológicas.

%\lettersection{Sobre mí}
Mi principal interés en el puesto tiene relación con aplicar los conocimientos
en \lsc{TDD} en línea a los últimos flujos de trabajo como los que practica su
empresa.

% 1. [ ] Por qué estoy interesado por el empleo o el trabajo que hace la empresa
% 2. [ ] Mencionar información del empleador o la posición (demuestra interés)
% 3. [ ] Mencionar habilidades específicas que te hacen encajar en el perfil
%        (Lo que puedes hacer por el empleador)
% 4. [ ] Comentar detalles relevantes del CV.

%\lettersection{¿Por qué esta empresa?}
Suspendisse commodo, massa eu congue tincidunt, elit mauris pellentesque orci,
cursus tempor odio nisl euismod augue. Aliquam adipiscing nibh ut odio sodales
et pulvinar tortor laoreet. Mauris a accumsan ligula.

% 1. [ ] Indicar que quiero la oportunidad de la entrevista por el puesto, o
%        el poder conversar con el empleador para aprender más de oportunidades
%        o planes de contratación.
% 2. [ ] Mencionar cómo haré el seguimiento, llamar al empleador, enviar mail...
% 3. [ ] Si estoy cerca ofrecesr visita y fecha.
% 4. [ ] Encantado de proveer información adicional.
% 5. [ ] Agradecimiento por la consideración.

%\lettersection{¿Por qué soy el indicado?}
Duis sit amet magna ante, at sodales diam. Aenean consectetur porta risus et
sagittis. Ut interdum, enim varius pellentesque tincidunt, magna libero sodales
tortor, ut fermentum nunc metus a ante.

\end{cvletter}


%-------------------------------------------------------------------------------
%                                  PIE DE PÁGINA 
%-------------------------------------------------------------------------------
\makeletterclosing[L]{Eduardo Javier Bray Gutiérrez}

\end{document}
