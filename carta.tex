%!TEX TS-program = xelatex
%!TEX encoding = UTF-8 Unicode

%-------------------------------------------------------------------------------
% CONFIGURATIONS
%-------------------------------------------------------------------------------
% 'a4paper' size by default, use 'letterpaper' for US letter
\documentclass[11pt, a4paper]{awesome-cv}

% Español
\usepackage[spanish]{babel}

% Configure page margins with geometry
\geometry{left=1.4cm, top=.8cm, right=1.4cm, bottom=1.8cm, footskip=.5cm}

% Specify the location of the included fonts
\fontdir[fonts/]

% If you would like to change the social information separator from a
% pipe (|) to something else replace \textbar
\renewcommand{\acvHeaderSocialSep}{\quad\textbar\quad}

%---------------------------------------

% Color for highlights
% Awesome Colors: awesome-emerald, awesome-skyblue, awesome-red,
%                 awesome-pink, awesome-orange, awesome-nephritis,
%                 awesome-concrete, awesome-darknight
\colorlet{awesome}{awesome-red}
% Uncomment if you would like to specify your own color
%\definecolor{awesome}{HTML}{CA63A8}

% Set false if you don't want to highlight section with awesome color
\setbool{acvSectionColorHighlight}{true}

% Colors for text
% Uncomment if you would like to specify your own color
%\definecolor{darktext}{HTML}{414141}
%\definecolor{text}{HTML}{333333}
%\definecolor{graytext}{HTML}{5D5D5D}
%\definecolor{lighttext}{HTML}{999999}


%-------------------------------------------------------------------------------
%	                                   HEADER
%-------------------------------------------------------------------------------
% !TEX encoding = UTF-8 Unicode
% !TEX root = ../curriculum.tex

%-------------------------------------------------------------------------------
%	                                HEADER SETUP
%-------------------------------------------------------------------------------
%	Comment any of the lines below if they are not required
% Add \\ inside the {} if you need to break a line, e.g., xing{your name\\}
% (The order of the entries is defined in awesome-cv.cls not here)
%-------------------------------------------------------------------------------

% Photo options: [circle|rectangle,edge/noedge,left/right]
%\photo[]{profile.png} 

% Personal information:
\name{Eduardo}{Bray G.}
%\familyname{familyname}% Optional
\position{
  Programador{\possep}
  Pedagogo{\possep}
  Músico}
\address{Macul, R.M. Santiago, Chile}
\mobile{(+56) 997-950-576}
\email{ejbray@uc.cl}
%\homepage{sitename}{url}

% Tech profile
\github{polirritmico}{}
%\gitlab{polirritmico}{}
%\stackoverflow{name}{}

% Professional profile
\linkedin{eduardo-bray}{}
%\xing{profile}{}
%\googlescholar{id}{name-to-display}{}
%\googlescholar{googlescholar-id}{}% Will add first and last name
%\medium{id}{}

% Social Networks
%\twitter{@polirritmico}{}
%\instagram{username}{}
%\skype{skype-id}{account}
%\pinterest{ebray187}{}
%\reddit{username}{}
\youtube{Youtube}{www.youtube.com/channel/UCWJ5DnWV2wTihLlw1fdxAxQ}
%\discord{username}{fullnumber-id}
%\tiktok{account}{}
%\snapchat{username}{}
%\facebook{username}{}
%\telegram{telegram}{test}

% Others:
%\wordpress{username}{}
%\blog{url}{blog name}
%\extrainfo{extra information}{}
%\extrainfo{extra information}{url}

%\quote{``Here you can add a meaningful quote or paragraph."}



%-------------------------------------------------------------------------------
%                              LETTER INFORMATION
%-------------------------------------------------------------------------------
%	All of the below lines must be filled out
%-------------------------------------------------------------------------------

% The company being applied to
\recipient
  {Equipo de recursos humanos}
  {Nombre compañía.\\Dirección\\Localidad, CP}
% The date on the letter, default is the date of compilation
\letterdate{\today}
% The title of the letter
\lettertitle{Solicitud de empleo para CARGO}
% How the letter is opened
\letteropening{Estimado Sr. Apellido,}
% How the letter is closed
\letterclosing{Atentamente,}
% Any enclosures with the letter
\letterenclosure[Adjunto]{Curriculum vitae}


%-------------------------------------------------------------------------------
%	                                  DOCUMENT
%-------------------------------------------------------------------------------
\begin{document}

% Add the HEADER with the defined personal information
% Give optional argument to change alignment(C: center, L: left, R: right)
\makecvheader[R]

% Print the footer with 3 arguments(<left>, <center>, <right>)
% Leave any of these blank if they are not needed
\makecvfooter
  {}
  {Eduardo Bray G.~~~·~~~Carta de presentación}
  {}

% Print the title with above letter informations
\makelettertitle

%-------------------------------------------------------------------------------
%	LETTER CONTENT
%	Each section is imported separately, open each file in turn to modify content
% If needed add \newpage between sections or a \vspace{X.Xmm} (can be negative)
%-------------------------------------------------------------------------------
\begin{cvletter}

%Párrafo de apertura: Declara la razón por la que escribes, cómo te enteraste del cargo o la organización. Añadir información básica sobre ti mismo.

Junto con saludar escribo esta carta con motivo del anuncio publicado en
LUGARDONDESEPUBLICA

%Segundo párrafo: Menciona por qué estás interesado por el empleo o escribe el trabajo que el empleador hace (decir simplemente que estás interesado no menciona las razones y puede parecer una carta-tipo). Demuestra que conoces suficiente del empleador o la posición para relacionar tus antecedentes con el empleador o la posición. Menciona habilidades específicas que te hacen encajar en las necesidades del empleador. (Enfócate lo que puedes hacer para el empleador, no en lo que el empleador puede hacer por ti.) Esta es una oportunidad para explicar con más detalle ítems relevantes de tu currículum. Refiérete al hecho que tu currículum está adjunto. Menciona otros adjuntos si son requeridos para la posición.

%Tercer párrafo: Indica que quieres la oportunidad de tener una entrevista por el puesto o que quieres conversar con el empleador para aprender más acerca de oportunidades o planes de contratación. Menciona lo que harás para hacer seguimiento, como llamar por teléfono al empleador en un par de semanas. Si vas a estar cerca de la ubicación del empleador podrías ofrecer una visita, indicar cuando. Indica que estarías encantado de proveer cualquier información adicional que se requiera. Agradece al empleador por su consideración.

\lettersection{Sobre mí}
Lorem ipsum dolor sit amet, consectetur adipiscing elit. Duis ullamcorper neque sit amet lectus facilisis sed luctus nisl iaculis. Vivamus at neque arcu, sed tempor quam. Curabitur pharetra tincidunt tincidunt.

\lettersection{¿Por qué esta empresa?}
Suspendisse commodo, massa eu congue tincidunt, elit mauris pellentesque orci, cursus tempor odio nisl euismod augue. Aliquam adipiscing nibh ut odio sodales et pulvinar tortor laoreet. Mauris a accumsan ligula.

\lettersection{¿Por qué soy el indicado?}
Duis sit amet magna ante, at sodales diam. Aenean consectetur porta risus et sagittis. Ut interdum, enim varius pellentesque tincidunt, magna libero sodales tortor, ut fermentum nunc metus a ante.

\end{cvletter}


%-------------------------------------------------------------------------------
% Print the signature and enclosures with above letter informations
%-------------------------------------------------------------------------------
%\makeletterclosing[L|C|R]{}. If empty uses @firstname and @lastname.
\makeletterclosing{Eduardo Javier Bray Gutiérrez}

\end{document}
