%!TEX TS-program = xelatex
%!TEX encoding = UTF-8 Unicode

%-------------------------------------------------------------------------------
% CONFIGURATIONS
%-------------------------------------------------------------------------------
% 'a4paper' size by default, use 'letterpaper' for US letter
\documentclass[11pt, a4paper]{awesome-cv}

% Español
\usepackage[spanish]{babel}

% Configure page margins with geometry
\geometry{left=1.4cm, top=.8cm, right=1.4cm, bottom=1.8cm, footskip=.5cm}

% Specify the location of the included fonts
\fontdir[fonts/]

% If you would like to change the social information separator from a
% pipe (|) to something else replace \textbar
\renewcommand{\acvHeaderSocialSep}{\quad\textbar\quad}

%---------------------------------------

% Color for highlights
% Awesome Colors: awesome-emerald, awesome-skyblue, awesome-red,
%                 awesome-pink, awesome-orange, awesome-nephritis,
%                 awesome-concrete, awesome-darknight
\colorlet{awesome}{awesome-red}
% Uncomment if you would like to specify your own color
%\definecolor{awesome}{HTML}{CA63A8}

% Set false if you don't want to highlight section with awesome color
\setbool{acvSectionColorHighlight}{true}

% Colors for text
% Uncomment if you would like to specify your own color
%\definecolor{darktext}{HTML}{414141}
%\definecolor{text}{HTML}{333333}
%\definecolor{graytext}{HTML}{5D5D5D}
%\definecolor{lighttext}{HTML}{999999}


%-------------------------------------------------------------------------------
%	                                HEADER SETUP
%-------------------------------------------------------------------------------
%	Comment any of the lines below if they are not required
% Add \\ inside the {} if you need to break a line, e.g., xing{your name\\}
% (The order of the entries is defined in awesome-cv.cls not here)
%-------------------------------------------------------------------------------

% Photo options: [circle|rectangle,edge/noedge,left/right]
%\photo[]{profile.png} 

% Personal information:
\name{Eduardo}{Bray G.}
%\familyname{familyname}% Optional
\position{
  Programador{\possep}
  Pedagogo{\possep}
  Músico}
\address{Macul, R.M. Santiago, Chile}
\mobile{(+56) 997-950-576}
\email{ejbray@uc.cl}
%\homepage{sitename}{url}

% Tech profile
\github{polirritmico}{}
\gitlab{polirritmico}{}
%\stackoverflow{name}{}

% Professional profile
\linkedin{eduardo-bray}{}
%\xing{profile}{}
%\googlescholar{id}{name-to-display}{}
%\googlescholar{googlescholar-id}{}% Will add first and last name
%\medium{id}{}

% Social Networks
%\twitter{@polirritmico}{}
%\instagram{username}{}
%\skype{skype-id}{account}
\pinterest{ebray187}{}
%\reddit{username}{}
\youtube{Youtube}{www.youtube.com/channel/UCWJ5DnWV2wTihLlw1fdxAxQ}
%\discord{username}{fullnumber-id}
%\tiktok{account}{}
%\snapchat{username}{}
%\facebook{username}{}
%\telegram{telegram}{test}

% Others:
%\wordpress{username}{}
%\blog{url}{blog name}
%\extrainfo{extra information}{}
%\extrainfo{extra information}{url}

%\quote{``Here you can add a meaningful quote or paragraph."}


%-------------------------------------------------------------------------------
%	                                  DOCUMENT
%-------------------------------------------------------------------------------
\begin{document}

% Add the HEADER with the defined personal information
% Give optional argument to change alignment(C: center, L: left, R: right)
\makecvheader[C]

% Print the footer with 3 arguments(<left>, <center>, <right>)
% Leave any of these blank if they are not needed
\makecvfooter
  {\today}
  {}
  {\thepage}

%-------------------------------------------------------------------------------
%	CV/RESUME CONTENT
%	Each section is imported separately, open each file in turn to modify content
% If needed add \newpage between sections or a \vspace{X.Xmm} (can be negative)
%-------------------------------------------------------------------------------

% Check the example file to learn how this works
%% !TEX encoding = UTF-8 Unicode
% !TEX root = ../cv.tex

% You have these options:

% Define a section for CV
% Usage: \cvsection{section-title}

% Define a subsection for CV
% Usage: \cvsubsection{subsection-title}

% Define a paragraph for CV
% Usage: \begin{cvparagraph}
%        \end{cvparagraph}

% Define an entry of cv information
% Usage: \cventry{position}{title}{location}{date}{description}
	% Define an environment for cvitems(for cventry)
	
	% Define a subentry of cv information
	% Usage: \cvsubentry{position}{title}{date}{description}

	% Define an environment for cvhonor
	% Define a line of cv information(honor, award or something else)

	% Usage: \cvhonor{position}{title}{location}{date}
	% Define an environment for cvskill
	% Define a line of cv information(skill)
	% Usage: \cvskill{type}{skillset}
	

%-------------------------------------------------------------------------------
%                Commands for elements of Cover Letter
%-------------------------------------------------------------------------------
% Define an environment for cvletter
% Define a section for the cover letter
% Usage: \lettersection{section-title}
% Define a title of the cover letter
% Usage: \makelettertitle
% Define a closing of the cover letter
% Usage: \makeletterclosing

% Section title
\cvsection{This is a Section Title inside a {\backslash}cvsection\{\}}

% Section content

\cvsubsection{This is a Subsection Title inside a {\backslash}cvsubsection\{\}}

\begin{cvparagraph}

This is a text inside a {\backslash}begin\{cvparagraph\}.

As you can see, you could put multiple paragraphs inside.

\end{cvparagraph}

\begin{cventries}
	% Usage: \cventry{position}{organization}{location}{date}{<description>}
  \cventry
    {position 1}
    {organization}
    {location}
    {date}
    {description}

  \cvextendentry
    {position 2}
    {date}
    {description}
    
  \cventry
    {position 3}
    {organization 2}
    {location}
    {date}
    {
      \begin{cvitems}
        \item {Item 1.}
        \item {Item 2.}
      \end{cvitems}
			\begin{cvsubentries}
			 	\cvsubentry{Subitem}{date}
			 	\cvsubentry{Subitem}{date}
			\end{cvsubentries}
      \begin{cvitems}
        \vspace{5.0mm}
        \item {Item 3.}
      \end{cvitems}
    }
\end{cventries}

%-------------------------------------------------------------------------------

% Section title
\cvsection{Skills}

% Section content
\begin{cvskills}

  \cvskill
    {Skill category} % Category
    {The details of the skill} % Skills
    
\end{cvskills}

%-------------------------------------------------------------------------------

% Section title
\cvsection{Honors}

\cvsubsection{International}

\begin{cvhonors}

  \cvhonor
    {Award} % Award
    {Event} % Event
    {Location} % Location
    {Date} % Date(s)

  \cvhonor
    {Award} % Award
    {Event} % Event
    {Location} % Location
    {Date} % Date(s)

\end{cvhonors}

% !TEX encoding = UTF-8 Unicode
% !TEX root = ../curriculum.tex

\cvsection{Resumen profesional}

\begin{cvparagraph}

Desarrollador, músico y pedagogo de la Pontificia Universidad Católica de Chile
con más de 15 años de experiencia en los rubros de enseñanza y producción
artística; con estudios de programación avanzada del Minor en Programación e
Informática en la misma casa de estudios.

Se ha desempeñado principalmente en áreas relacionadas a la tecnología
integrando soluciones de software y hardware a obras artísticas y la
construcción de controladores MIDI. Tiene más de 15 años de experiencia en
entornos \lsc{GNU}/Linux (Gentoo, Mint, OpenSuse, entre otros), mantenimiento
de PC (software y hardware), asistencia a usuarios y uso avanzado de diferentes
programas como LibreOffice, Gimp, Inkscape y Godot. En cuanto a programación ha
trabajado en lenguajes como C, C\#, GDscript, LaTeX, Python, CSS, Bash, XHTML,
entre otros.

\end{cvparagraph}


% !TEX encoding = UTF-8 Unicode
% !TeX root = ../curriculum.tex

\cvsection{Educación}

\begin{cventries}

%---------------------------------------------------------
  \cventry
    {Licenciado en Educación} % Degree
    {PUC (Pontificia Universidad Católica de Chile)} % Institution
    {Santiago, Chile} % Location
    {MAR 2015 - ENE 2016} % Date(s)
    {
      \begin{cvitems} % Description(s) bullet points
        \item {Estudio de seminario sobre el aprendizaje musical de los alumnos a través de las Tecnologías de Información y Comunicación.}
        \item {Profesor de Educación Media en Educación Musical.}
        \item {Aprobado con distinción.}
      \end{cvitems}
    }

%---------------------------------------------------------
  \cventry
    {Licenciado en Música} % Degree
    {PUC (Pontificia Universidad Católica de Chile)} % Institution
    {Santiago, Chile} % Location
    {MAR 2005 - JUL 2014} % Date(s)
    {
	    \begin{cvitems} % Description(s) bullet points
	     	\item {Intérprete superior de Percusión.}
	     	\item {Estudios técnicos en programación avanzada del Minor en Programación e Informática.}
	     	\item {Tesorero y Vicepresidente del Centro de Estudiantes de Música (\lsc{CEEM}).}
	    \end{cvitems}
    }

%---------------------------------------------------------
\end{cventries}

% !TEX encoding = UTF-8 Unicode
% !TEX root = ../curriculum.tex

\cvsection{Habilidades}

\begin{cvskills}

%---------------------------------------------------------
  \cvskill
    {Programación} % Category
    {C, Python, C\#, C++, Bash, GDScript, RegEx, LaTeX, EPUB, Markdown.} % Skills

%---------------------------------------------------------
  \cvskill
    {Librerías y herramientas} % Category
    {GNU/Linux, Gentoo, Archlinux, Ableton Live, Max, LibreOffice, WPF, Neovim, KDE, QT5 Designer,} % Skills
    %Beautifull Soup, VMVirtual, TexStudio, Make,
  \cvskill
    {} % Category
    {Make4ht, GUT, Tidy, Perl scripts, Sed, Beautifull Soup, VM Virtual Box, TexStudio, Make, Sigil.} % Skills

%---------------------------------------------------------
  \cvskill
    {Gestión de Proyectos} % Category
    {Metodología Kanban, GIT, Trello, Google Workspace, Git Hooks.} % Skills

%---------------------------------------------------------
  \cvskill
    {Idiomas} % Category
    {Español (nativo) e Inglés (fluído).} % Skills

%---------------------------------------------------------

\end{cvskills}

\input{sections/experiencia-informatica.tex}
% !TEX encoding = UTF-8 Unicode
% !TEX root = ../curriculum.tex

\cvsection{Actividad Extracurricular}


%---------------------------------------------------------
%	CONTENT
%---------------------------------------------------------
\begin{cventries}

%---------------------------------------------------------
  \cventry
    {Varias comunidades} % Affiliation/role
    {Editor de libros electrónicos} % Organization/group
    {Int.} % Location
    {2013 - Presente} % Date(s)
    {
      \begin{cvitems} % Description(s) of experience/contributions/knowledge
        \item {Digitalización de libros y corrección automatizada de procesos \lsc{OCR}.}
        \item {Maquetación de libros en formato \lsc{EPUB}, \lsc{XHTML} mediante \lsc{CSS} en Sigil.}
        \item {Confección de herramientas y plugins para facilitación de los procesos de corrección (Python y PyQT5).}
        \item {Videos tutoriales de Inkscape para la vectorización de imágenes.}
      \end{cvitems}
    }

%---------------------------------------------------------
  \cventry
    {Particpación como usuario de software open source} % Affiliation/role
    {Grupos y foros GNU/Linux} % Organization/group
    {Int.} % Location
    {2005 - Presente} % Date(s)
    {
      \begin{cvitems} % Description(s) of experience/contributions/knowledge
        \item {Reporte de bugs y creación de parches diff.}
        \item {Asesoría y soporte a usuarios de sistemas Gentoo Linux.}
        \item {Publicación de tutoriales y configuraciones.}
      \end{cvitems}
    }


%---------------------------------------------------------
\end{cventries}

%\input{sections/escritura.tex}


%-------------------------------------------------------------------------------

\end{document}
